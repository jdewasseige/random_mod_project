\section{Conclusion} % (fold)
\label{sec:conclusion}

To conclude this work on the game \monopoly~a critical evaluation of the results and its 
implications is necessary. From our research on previous work we are convinced that the 
simplifications we did to model the game do not really influence the results. 

The invariant state that was computed does fit to previous works and it is legitimate to say 
that it is correct. Some further hypotheses were based on this. 
We can state that some streets are more lucrative than other streets because of the 
probability to land on these fields and due to a better ration of cost and rent.

We should conclude that this work is a computation of the stationnary probabilities which can 
be used to deduce a strategy to determine which streets are more favorable to buy than others.
\todo{need a paragraph about the second part - modeling some stopping time}
Our results rely on playing for a considerably long time because our calculations derive from 
stationnary probabilites. If you play one game of monopoly and you wonder why having bought the ``best'' 
streets according to our analysis does not make you the winner, keep in mind that we are dealing 
with a model where the number of turns being played goes to infinity. So you should play a couple 
more times to get a confirmation of our results.

\monopoly~is a good application of \textsc{markov chains}, because on the one hand it gives you results
that are easy to understand and on the other hand and to better
understand how \textsc{markov chains} can be applied to real world situations.
\todo{Further work can contain estimations for 
the duration of a game, depending on different strategies - WE NEED A RESULT THAT DOES THAT}. 

% section conclusion (end)
