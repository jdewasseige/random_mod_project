\section{Conclusion} % (fold)
\label{sec:conclusion}

To conclude this work on the game \textit{Monopoly} a critical evaluation of the results and its 
implications is necessary. From our research on previous work we are convinced that the 
simplifications we did to model the game do not really influence the results. 

The invariant state that was computed does fit to previous works and it is legitimate to say 
that it is correct. Some hypotheses were made based on this. 
We can state that some streets are more efficient to win the game than others because of the 
probability to come to those action fields for the other players and due to a lower price of
purchase or a higher price of rent.


Finally, it is important to state that this work is a computation of the probabilities which can 
be used to deduce a strategy to determine which streets are more favorable to buy than others. In no
case this work represents a simulation of a game. If you only play once any other strategy is likely 
to be as successful or even better than the one you deduce from these results. Only in the long term, 
the advised streets are likely to turn out to be more efficient.

\textit{Monopoly} is a good application to see \textsc{Markov-Chains} in action and to really 
understand how this concept works for an application. Further works can contain estimations for 
the duration of a game, depending on different strategies. because the strategies that were 
chosen in this paper are simplified and do not represent real players' behavior. 

% section conclusion (end)
