\section{Understanding the code} % (fold)
\label{sec:understanding_the_code}
Now that we detailed the model and how and why we built it so,
it seems a good idea to give some insight on how we obtained
the result that will be presented in the next section.
This section will therefore allow us to clearly \emph{explain the code}
that we used throughout the entire modeling process.

\paragraph{Filling in the transition matrices} % (fold)
\label{par:filling_the_transition_matrices}
\addcontentsline{toc}{subsection}{Filling in the transition matrices}
As explained from a mathematical point of a view 
in~\ref{sub:modeling_using_markov_chains}, we need to fill in the
\emph{transition matrices} that will represent the probabilities
of going from one state to another throughout the board.
Note that we need to fill two matrices, one in the case
we choose to wait and get a double in jail and one in the
case we directly pay a fine.
Firstly the method \lstinline|fillMatrixWithDiceProb| first fills the matrix
with the basic dice sum probabilities (e.g. from state $i$ there is a
$1/36$ probability to go to state $i+2$, $2/36$ to state $i+3$ and so
forth).
Next we continue filling in the matrices by taking into account
the \emph{Chance} and \emph{Community chest} cards which both have
some cards redirecting to some other fields (i.e. \emph{Chance} 10
out of 17 cards and \emph{Community chest} 2 out of 17).
This is done in \lstinline|fillMatrixWithActionCommunityCardsProb|.
Finally, we take into account the strategies fine and double throw,
respectively in the methods \lstinline|fillMFineWithJailProb|
and \lstinline|fillMDoubleThrowJailProb|.

% paragraph filling_the_transition_matrices (end)

\paragraph{Getting the stationary probabilities} % (fold)
\label{par:getting_the_stationary_probabilities}
\addcontentsline{toc}{subsection}{Getting the stationary probabilities}
Once the transition matrix is filled up, we can easily
get the stationary probabilities, which we need to know
which fields are worth \enquote{betting on}.
Instead of solving it using the classical formula $\pi P = \pi$,
or iterating through a few step using a starting vector $\pi_0$,
we decided it was more efficient and simple to do it using the 
\lstinline|markovChain| class of the \lstinline|discreteMarkovChain| module.
We then get the stationary probabilities in the form of a vector
with our method \lstinline|calculateStationaryVector|.
To get the above mentioned function running we need a package.
For that, run in command line
\begin{center}
  \lstinline|pip install discreteMarkovChain|
\end{center}
or use \lstinline|pip3| depending on your system.
% paragraph getting_the_stationary_probabilities (end)

\paragraph{Writing data to \texttt{.csv} files} % (fold)
\label{par:writing_data_to_csv_files}
\addcontentsline{toc}{subsection}{Writing data to \texttt{.csv} files}
One will note the presence of a \lstinline|writeDataToCsvFile|
method. It make it possible to write the data from the stationary probability
vectors into two \texttt{.csv} files and later it enables us to use those
files with \LaTeX to get the tables~\ref{tab:statio_double}
and~\ref{tab:statio_double}, both presented in result section~\ref{sec:results}.
% paragraph writing_data_to_csv_files (end)


% section understanding_the_code (end)