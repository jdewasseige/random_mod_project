\section{Results} % (fold)
\label{sec:results}

\subsection{Stationnary probabilites} % (fold)
\label{sub:stationnary_probabilites}

First, it needs to be said that since our markov chain is homogenous, 
the transition probabilities do not depend on time (the turn in our case).
Also, it can be seen that the markov chain is irreducible and positive recurrent guaranteeing 
that all states have a finite hitting time. 
Thus, we know that a stationnary distribution exists and we can calculate it.

The stationary probabilities are obtained using
the \lstinline|discreteMarkovChain| package and
the results for both strategies are shown in
tables~\ref{tab:statio_double} and~\ref{tab:statio_fine}.

\newpage
\begin{changemargin}{-0.5cm}{2cm} % stationary probabilities tables (fold)
  {
  \noindent\centering
    \begin{minipage}{0.7\linewidth}
      
      \begin{tabular}{|c|c|}%
         \hline
         \bfseries State & \bfseries Stationary probability % specify table head
         \csvreader[head to column names]{../code/csvFiles/statio_proba_double.csv}{}% use head of csv as column names
         {\\\hline \state &$\proba~\%$}
         \tabularnewline \hline
     \end{tabular}
     \begin{flushleft}
       \captionof{table}{\emph{Double} strategy.}
       \label{tab:statio_double}
     \end{flushleft}
    \end{minipage}
    \begin{minipage}{0.7\linewidth}
       \begin{tabular}{|c|c|}%
        \hline
        \bfseries State & \bfseries Stationary probability % specify table head
        \csvreader[head to column names]{../code/csvFiles/statio_proba_fine.csv}{}% use head of csv as column names
        {\\\hline \state &$\proba~\%$}
        \tabularnewline \hline
    \end{tabular}
     \begin{flushleft}
       \captionof{table}{\emph{Fine} strategy.}
       \label{tab:statio_fine}
     \end{flushleft}
    \end{minipage}
  }
  %\vspace{0.7cm}
\end{changemargin} % stationary probabilities tables(end)

% subsection stationnary_probabilites (end)

\subsection{Most lucrative streets to buy} % (fold)
\label{sub:most_lucrative_streets_to_buy}

Whenever we play \monopoly~we have to ask us at one point whether we want to buy or not a certain street 
that we landed on. Those who have played \monopoly~quite often know that there is that common sense that 
says: ``Buy the orange streets - They are the best!'' 
and ``Never buy the electricity or water factory - They are not any good''. 
Meant by the orange streets are the three streets on the spots 16,18 and 19. The electricity and water 
factories are the spots 12 and 28 respectively.

To start with, we briefly want to explain how you can buy streets, what you can do with it and how you can 
make money with it. Whenever a player lands on a \emph{street}, which is either a \emph{a colored street}, 
\emph{a train station} or \emph{a factory} that is not bought yet by another player, he can buy the street 
from the bank for a certain amount of money that we define at this point as the  \textbf{buying cost}.
Once the street is bought, the owner of the street earns money whenever another player lands on that street. 
This rent is defined as the \textbf{simple rent}. 
Once a player owns all colors of a \emph{a type of colored street} or all \emph{train stations} or all 
\emph{factories}, he will earn more than the simple rent. When owning all colors of a \emph{a type 
of colored street}, the player has the ability to built houses for a certain price to increase the rent other 
players need to pay when landing on the street. He can built up to 5 houses (= 1 hotel) per colored street 
for a cost that we define at this point the \textbf{hotel cost} and will get the \textbf{hotel rent}.
On \emph{train stations} and \emph{factories} we are not allowed to built any houses so the best possible 
rent is the rent one earns when owning all of them. For the sake of simplicity, we will call the cost of 
buying all streets also the hotel cost and the resulting rent also the hotel rent.

Now, we want to try to proof or contradict the statements written above by \emph{calculating the average 
amount of turns one player has to wait until the cost of having bought the street pays off by the rent he 
gets} which we define as the \textbf{number of turns to equalize rent and cost}. For this we will use the stationnary probability per field, the cost and the rent. 
The values for cost and rent we used for our calculation can be found on the officiel wikipage for monopoly 
~\cite{wiki:monopolyRules}. Again, the rule 
that a player can throw the dice again if he throws a double is neglected for this analysis, so that we get 
the following formula to calculate the number of turns with $\pi_{stat}$ being the stationnary distribution.
\[
\text{Turns} = \frac{\text{Cost}}{\pi_{stat} \times \text{Rent}}
\]
We will consider two scenarios:
\begin{itemize}
	\item Cost = buying cost and rent = simple rent
	\item Cost = hotel cost and rent = hotel rent
\end{itemize}

Let's start we our first scenario.
We calculate number of turns to equalize rent and cost in table~\ref{tab:turns_simple} 
and then we order this table in ascending order showing the ``best'' field on top. 
Certain field have a value of infinity because they are not up for sale. We included these fields for 
reasons of readability and they are naturally not included in
the ordered table~\ref{tab:ordered_turns_simple}.

Looking at spot number 15, we can see that in order to equalize the cost of this train station, a player 
has to wait on average until the dice has thrown 298 times by other players.
We can see directly that the best streets are the factories, whereas the orange streets are not remarkably 
better than other streets. Does that mean that the common sense for \monopoly~is completely wrong?
Let's have a look at our second scenario in tables~\ref{tab:turns_hotels} and~\ref{tab:ordered_turns_hotels}.
Now we see what we expected to see. The best streets are the orange streets and the factories are the worst 
fields. In this scenario it is also remarkable how much less turns are needed to equalize the cost for 
\emph{hotels} on colored streets in comparison to the turns needed to equalize the cost for buying factories. And 
how much less turns in general are needed to equalize the costs in comparison to the first scenario.

What can we \emph{conclude}? 
It is obvious that players are never paying off the cost of buying a street in the beginning without 
building hotels. It just takes too much time (ca. 300 turns) and the rent is too small. However, once the 
players have built hotels the rent gets much more important since it is paying off directly (ca. 28 turns). 
So a first result is that the hotel rent is ``10 times better'' than the simple rent.
Having the first result in mind, we also conclude that the order of the order of the ``best streets'' in 
the second scenario is the order a player has to look at when buying a street and thus, we can confirm 
the ``common sense'' of \monopoly~players. 

\begin{changemargin}{-0.5cm}{2cm} % nb_turns simple and hotels  (fold)
  {
  \noindent\centering
    \begin{minipage}{0.7\linewidth}
      
      \begin{tabular}{|c|c|}%
         \hline
         \bfseries State & \bfseries Number of turns % specify table head
         \csvreader[head to column names]{../code/csvFiles/turns_simple.csv}{}% use head of csv as column names
         {\\\hline \state & \nbTurns}
         \tabularnewline \hline
     \end{tabular}
     % \begin{flushleft}
       \captionof{table}{Number of turns to equalize \\\emph{simple} rent and \emph{buying} cost.}
       \label{tab:turns_simple}
     % \end{flushleft}
    \end{minipage}
    \begin{minipage}{0.5\linewidth}
        \begin{tabular}{|c|c|}%
           \hline
           \bfseries State & \bfseries Number of turns % specify table head
           \csvreader[head to column names]{../code/csvFiles/turns_hotels.csv}{}% use head of csv as column names
           {\\\hline \state & \nbTurns}
           \tabularnewline \hline
       \end{tabular}
     %\begin{flushleft}
       \captionof{table}{Number of turns to equalize \\\emph{hotel} rent and \emph{hotel} cost.}
       \label{tab:turns_hotels}
     %\end{flushleft}
    \end{minipage}
  }
  \vspace{0.7cm}
\end{changemargin} % nb_turns simple and hotels (end)


\begin{changemargin}{-0.2cm}{2cm} % nb_turns simple  (fold)
  {
  \noindent\centering
    \begin{minipage}{0.61\linewidth}
      \begin{center}
         \begin{tabular}{|c|c|}%
            \hline
            \bfseries Rank & \bfseries State % specify table head
            \csvreader[head to column names]{../code/csvFiles/ordered_fields_simple_turns.csv}{}% use head of csv as column names
            {\\\hline \rank & \state}
            \tabularnewline \hline
        \end{tabular}
        % \begin{flushleft}
          \captionof{table}{\emph{Ordered} fields by number\\ of turns needed to equalize \emph{simple} 
		  \\rent and \emph{buying} cost.}
          \label{tab:ordered_turns_simple}
      \end{center}
     
     % \end{flushleft}
    \end{minipage}
    \begin{minipage}{0.5\linewidth}
      \begin{center}
           \begin{tabular}{|c|c|}%
              \hline
              \bfseries Rank & \bfseries State % specify table head
              \csvreader[head to column names]{../code/csvFiles/ordered_fields_hotels_turns.csv}{}% use head of csv as column names
              {\\\hline \rank & \state}
              \tabularnewline \hline
          \end{tabular}
        %\begin{flushleft}
          \captionof{table}{\emph{Ordered} fields by number of turns needed to equalize
		  \emph{hotel} rent\\ and \emph{hotel} cost.}
          \label{tab:ordered_turns_hotels}
      \end{center}
        
     %\end{flushleft}
    \end{minipage}
  }
  \vspace{0.7cm}
\end{changemargin} % nb_turns simple (end)

\newpage

\subsection{Simulating} % (fold)
\label{sub:simulating}
Using the file \texttt{simulation.py} as explained in previous section,
we are able to run one or more games of two players. 
We will play a maximum number of 1000 turns and will get one of three results:

\begin{itemize}
	\item player 1 wins,
	\item player 2 wins or
	\item there is a draw between player 1 and player 2.
\end{itemize}

\noindent We start the simulation under the following conditions:

\begin{enumerate}
	\item Neither of the two players are able to buy any of the streets or able to build houses on 
	streets they already posses.
	\item The players will not lose or win money when drawing a \emph{chance}- or 
	\emph{community chest} card.
	\item Player 1 owns the orange streets with hotels and all the train stations
	and player 2 owns the brown streets, the light blue streets and the dark blue streets
	with hotels.
\end{enumerate}

\noindent Now we have two options to simulate the game: We can use our model using the 
\emph{Wait and get double strategy} or the \emph{Pay fine strategy} as explained 
in subsection~\ref{sub:modeling_using_markov_chains}.


\noindent Let's see a simulation of the game using the \emph{Pay fine strategy}:

\vspace{1\baselineskip}

\begin{center}
  \texttt{
  Fields after step  0 (1 and 2) :  5  and  6 \\
  Money after step 0 (1 and 2)  :  2000  and  2000	\\		
  -------------------------------------------------------\\
  Fields after step  1 (1 and 2) :  10  and  13\\
  Money after step 1 (1 and 2)  :  2000  and  2000\\
  -------------------------------------------------------\\
  Fields after step  2 (1 and 2) :  16  and  21\\
  Money after step 2 (1 and 2)  :  2000  and  2000\\
  -------------------------------------------------------\\
  Fields after step  3 (1 and 2) :  23  and  28\\
  Money after step 3 (1 and 2)  :  2000  and  2000\\
  -------------------------------------------------------\\
  ...\\
  -------------------------------------------------------\\
  Fields after step  42 (1 and 2) :  28  and  33\\
  Money after step 42 (1 and 2)  :  830  and  6120\\
  -------------------------------------------------------\\
  Fields after step  43 (1 and 2) :  33  and  1\\
  Money after step 43 (1 and 2)  :  830  and  6320\\
  -------------------------------------------------------\\
  Fields after step  44 (1 and 2) :  39  and  7\\
  Money after step 44 (1 and 2)  :  -1170  and  8320\\
  -------------------------------------------------------\\
  Player 1 is broke and the game ended after 88 turns.\\
  The final money distribution is: \\
  ---------------| Player 1: 0  | Player 2: 8320  |-------------}
\end{center}

\vspace{1\baselineskip}

Since simulating only one game is not very representive to decide which 
player has the better initial starting conditions, we also implemented 
a function \lstinline|getStatisticForSimulation| that allows us to get a statistic 
when running the game a thousand or so times. 

After 1000 games playing the game with the \emph{Wait and get double strategy}, we 
get:

\vspace{1\baselineskip}

\begin{center}
  \texttt{------------| Player 1 wins | Player 2 wins | draw  |-------------\\
------------|  0.384        | 0.246         | 0.370 |-------------}
\end{center}

\vspace{1\baselineskip}

and when playing the \emph{Pay fine strategy.} in the case of 1000 games and a \enquote{fine} strategy
we get:

\vspace{1\baselineskip}

\begin{center}
  \texttt{------------| Player 1 wins | Player 2 wins | draw  |-------------\\
------------| 0.465          | 0.221        | 0.314 |--------------}
\end{center}

\vspace{1\baselineskip}

We can see that player one has the better starting conditions even if it was player two who
won our single game. The \emph{Wait and get double strategy} seems to support draws between the
two players which is not surprising since it slows down the game because players stay longer in jail.
% subsection simulating (end)


% subsection most_lucrative_streets_to_buy (end)

% section results (end)