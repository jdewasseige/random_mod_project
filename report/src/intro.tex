\section{Introduction} % (fold)
\label{sec:introduction}

During the course of Random-Modeling, we saw the abstract concepts of how to treat and predicts the 
outcome of events of a random nature. Examples for events and variables that can be modeled by the 
concepts taught in the course are all kind of games that depends on luck, such as a coin toss, 
risk-calculation for insurance company and other events with an uncertain outcome. Since the course
is largely focused on the theoretical approach, we wanted to use \textit{Markov-Chains} to model 
a real problem. This mathematical structure is widely used in risk-modeling and was therefore the
most convenient for us. 

As a subject for the project we chose the table game \textsc{Monopoly} because it is well known and 
there some work has already been done so we could rely on previous works. Another advantage of 
\textsc{Monopoly} lies in the simplicity of the game's rules. The is a very limited list of rules, 
because the main complexity comes from the diversity of different action fields. Therefore it is not
very difficult to model the players' strategies. You easily define behaviors, so the only part to 
be predicted is the action field on which a player arrives according to the dices. The throw of 
dices is random variable that can be perfectly modeled by \textit{Markov-Chains}.

This paper aims to compute the efficiency of the different action fields than can be ``bought'' by 
players. There is also an estimation of the duration of the game given that one player owns a number
of action fields and another player which has a limited amount of money and no right to buy anything.


% section introduction (end)
