\section{Introduction} % (fold)
\label{sec:introduction}

During \textsc{Introduction to random modeling}, we saw abstract concepts of how to treat and predict the 
outcome of events of a random nature. Examples for events and variables that can be modeled by the 
concepts taught in the course are all kind of games that depend on chance, such as a coin toss, 
risk-calculation for insurance companies and other events with an uncertain outcome. Since the course
is largely focused on the theoretical approach, we used \textsc{Markov chains}	 to model 
a real problem. This mathematical structure is widely used in risk-modeling and was therefore the
most convenient for us. 

As a subject for the project we chose the table game \monopoly~because it is a well known board game and 
due to already existing papers on this model, we are able to compare our results. 
Another advantage of \monopoly~lies in the simplicity of the game's rules. 
There is a limited list of rules so that the main complexity does not come from a large number 
of choices the player has on his turn but from the diversity of different action fields. Therefore it is not
very difficult to model the players' strategies. 
The only ``random'' parts are therefore the action field on which a player lands according to the dices
and an action card me might draw. 
Both things can perfectly be modeled by different transition probabilities from one state to another.

This paper aims to compute a ranking of the different streets that can be bought according to their rentability. 
In the final part we created a simplified simulation based on the probabilities given by the \textsc{Markov 
chains}.
It simulates a couple of games with a set of initial conditions. In this manner it is possible to forecast
who is more likely to win depending on which street you own.

This report will lead you through our work by first \emph{introducing} the board
game, explaining the \emph{hypothesis} we decided to apply and how it
was \emph{modeled} using Markov chains, all of this in section~\ref{sec:model}.
Then, the reader will find in section~\ref{sec:understanding_the_code}
detailed explanation on how the \emph{code} works and how it relates
to what we want to find in the results.
Next the \emph{results} are presented in section~\ref{sec:results}.
In this section, the most important results we obtained will be 
presented using tables representing the fields of the board
and those will be analyzed.
Finally, the last section~\ref{sec:conclusion}, will allow us
to draw a \emph{conclusion} and think about \emph{further work}.

% section introduction (end)
