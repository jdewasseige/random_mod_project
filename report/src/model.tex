\section{Model} % (fold)
\label{sec:model}

\subsection{The board game} % (fold)
\label{sub:the_board_game}
The board game \monopoly~exists since the Great Depression and has
become so popular that it has been translated into more than 37
languages and is the best-selling board game in the world.
The original idea was \enquote{\textit{to demonstrate that an economy which
rewards wealth creation is better than one in which monopolists work
under few constraints[1] and to promote the economic theories of Henry
George and in particular his ideas about taxation}}~\cite{wiki:monopoly}.

For those who don't know the board game \monopoly, we want to briefly 
explain the rules at this point. 
The game is played on a board having 40 different ``action fields'' we the player 
can or has to perform an action. During the game the players travel around the board using 
two dices as their mean to go forward. 
In the  beginning of the game the players are given a certain amount of money which the player 
can or has to use to perform an action on the ``action field'' consisting of buying the ``action field''
as their personal property or paying diverse fees. 
The goal of the game is to make as much money as possible and to finally get all the others players money
so that one is the only player that is not broke. 
% subsection the_board_game (end)

\subsection{Hypothesis and simplifications} % (fold)
\label{sub:hypothesis_and_simplifications}
For the sake of this project, we are going to use a simplified
version of \monopoly. The simplifications we make will be described throughout
this section as well as an explanation that the game is not affected significantly by 
these simplifications.

\paragraph{Removing the rule: ``Additional turn when throwing a dice double''} % (fold)
\label{par:removing_special_rules_with_doubles}
The rules state that when a player throws a double, he has an additional turn and can throw the dice 
again. But, throwing three doubles in a row forces the player to go to jail.
The rule stating that one can get out of jail by trying to throw a double is not removed.
Removing this rule for the sake of simplicity will not really affect the game outcome, since
throwing the dice twice in one turn is very similar to throwing the dice twice in two turn in 
terms of modeling.
% paragraph removing_special_rules_with_doubles (end)

\paragraph{Simplifying jail outcomes} % (fold)
\label{par:simplifying jail}
While modeling the Markov process, it quickly appeared that the \jail~field
may cause some issues as some other factors may lead to going there (chance
and community chests cards) and getting out also leaves 3 possibilities.
Indeed the player can~\cite[p.18]{mit_course_notes}
\begin{enumerate}
  \item wait in jail for three turns to get out,
  \emph{  
  \item try to throw a double to get out,
  }
  \emph{
   \item pay a 50 euros fine to get out, or
  }
  \item use a \enquote{Get Out of Jail Free} card.
\end{enumerate}
\todo{ I would delete this part:
Intuitively, the first strategy seems to be the best one to play once a lot
of buildings are around the game but the 3rd one seems best at the beginning
since it is essential to quickly get some streets.
}

The fact that four different strategies exist and that they are time related
adds a lot of complexity and we therefore decided to make the assumption that
only two choices are available for the player,
those being the \emph{second} and the \emph{third}.
Looking at the low probabilities of the events to \emph{to wait in jail for three turns}
and \emph{use a \enquote{Get Out of Jail Free} card} we can remove these possibilities without 
any concerns.

% paragraph simplifying_the_jail_related_possibilities (end)

% subsection hypothesis_and_simplifications (end)

\newpage
\subsection{Modeling the \emph{board} using Markov chains} % (fold)
\label{sub:modeling_using_markov_chains}
As seen in class, a Markovian process is an appropriate way of representing a process
that has no memory and where the probability of being in a state at time $n$
only depends on our state at time $n-1$.
In our model a \emph{time unit} is represented by one dice throw
per player (one would normally say one turn but
we assumed in section~\ref{sub:hypothesis_and_simplifications} that no
re throw was allowed).
The game will thus be represented using 40 states $(X_n)_{0\leq n\leq39}$,
each state being one the fields. 
To conclude, it can be seen that our model satisfies all the conditions for 
a markov chain since $E(X_n | X_{n-1},...,X_0) = E(X_n | X_{n-1})$.
At this point, we can also notice that the markov chain is a \emph{homogenous chain}: 
$\mathbb{P}(X_n = x | X_{n-1} = y) = \mathbb{P}(X_{n+1} = x| X_{n} = y)$ with $x,y \in \text{States}$.
To put it in words that means that, the probablity of getting form state x to state y is the same at every 
time during the game and does not depend on the state the player has been before having been at state x.

By setting the \go~field to be state 0, we get the following \emph{transition matrix}
\[
  P = \begin{pmatrix}
    p_{00} & p_{01} & \dots \\
    p_{10} & p_{11} &  \\
    \vdots      &  & \ddots
  \end{pmatrix}
\]

\begin{enumerate}
  \item The \emph{dice sum} (see table~\ref{tab:dice_sum_prob}) is the 
  most decisive influence for the transition probabilities
  as it accounts for a majority of the fields.
  \item Next, one can take into account the chance and community chest
  fields as some of them force the player to change their current location.
  \item Finally, we need to consider two other special fields :
  the Jail and GoToJail. The latter has a probability
  of one to immediately go to Jail.
\end{enumerate}

\begin{table}
  \begin{center}
    \begin{tabular}{|c|c|}
      \hline
        \textbf{Dice Sum} & \textbf{Probability} \\
        \hline \hline
        2, 12 & 1/36 \\
        \hline
        3, 11 & 2/36 \\
        \hline
        4, 10 & 3/36 \\
        \hline
        5, 9 & 4/36 \\
        \hline
        6, 8 & $5/36$ \\
        \hline
        7 & 6/36 \\
      \hline
    \end{tabular}
  \end{center}
  \caption{Probability distribution of a two dice sum.}
  \label{tab:dice_sum_prob}
\end{table}

In addition we have to take into account the \emph{Community Chest Cards} and 
the \emph{Action Cards}. \todo{Explain how that influences the probabilities.
We have 10/17 action cards that make the player move to different fields and 
2/17 community chest cards that make the player move. 
The following websites are used: \url{http://monopoly.wikia.com/wiki/Community_Chest 
and http://monopoly.wikia.com/wiki/Chance}}

\todo{Explain somewhere that since the transition probabilities does not depend on
time - THIS IS EXPLAINED SINCE THE CHAIN IS HOMOGENOUS  --, that the Markov chain is irreducible and positive recurrent,
all states are guaranteed to have a finite hitting time and thus
stationary probabilities exist.}

% \textbf{Defining a richness matrix}
% At this point we need to think about how we will manage the changing
% banks account of each player. 
% Ideas:
% - a matrix telling what each field is worth
% - simulating many games using stationary probas

\newpage
\begin{figure}[H]
  \begin{center}
    \begin{tikzpicture}[font=\small\sffamily,label=monopoly]
  \tikzset{my_state/.style={state, 
                                     minimum width=1.3cm,
                                     line width=0.13mm,
                                     fill=gray!20!white}}
  \tikzset{my_state2/.style={state, 
                                     minimum width=1.7cm,
                                     line width=0.43mm,
                                     fill=gray!60!white}}
                                     
  % Go
  \node[my_state2, ] (go) {Go};  
                                    
  % Denmark                                 
  \node[my_state, above right = 1 of go] (den) {Den};
  \node[my_state, right = 1 of den] (airport1) {Air1};
  % Belgium
  \node[my_state, right = 1 of airport1] (bel) {Bel};
  \node[my_state2, below right = 1 of bel] (jail) {Visit};
  % Netherlands
  \node[my_state, below right = 1 of jail] (neth) {Neth};

  \node[my_state, below = 1 of neth] (airport2) {Air2};
  % Spain
  \node[my_state, below = 1 of airport2] (spain) {Sp};

  \node[my_state2, below left = 1 of spain] (park) {Parking};
  % United Kingdom
  \node[my_state, below left = 1 of park] (uk) {UK};

  \node[my_state, left = 1 of uk] (airport3) {Air3};
  % Italy
  \node[my_state, left = 1 of airport3] (ita) {It};

  \node[my_state2, above left = 1 of ita] (gotojail) {GoToJail}; 
  % France
  \node[my_state, above left= 1 of gotojail] (fr) {Fr};

  \node[my_state, above = 1 of fr] (airport4) {Air4};
  % Germany
  \node[my_state, above = 1 of airport4] (ger) {Ger};

  % 3 other coins
  % chances and community chests
    % \node[my_state, above right = 1 of ] (chance) {Chance1};
    % \node[my_state, above right = 1 of ] (chance2) {Chance2};
    % \node[my_state, above right = 1 of ] (chance3) {Chance3};
    % \node[my_state, above right = 1 of ] (chest1) {Chest1};
    % \node[my_state, above right = 1 of ] (chest2) {Chest2};
    % \node[my_state, above right = 1 of ] (chest3) {Chest3};
  \node[my_state, above right = 3 of gotojail] (jailStrat) {See fig.~\ref{fig:jail_double} and~\ref{fig:jail_pay}};
  
 \draw[every loop, auto=right, line width=0.3mm, >=latex, draw=red, fill=black, bend right]
 
 (go) edge[bend left]
   node {} (den)
 (go) edge[bend left=71]
   node {} (airport1)
 (go) edge[]
   node {} (bel)
   
 (den) edge[draw=blue, bend left]
   node {} (airport1)
 (den) edge[draw=blue, bend left=40]
   node {} (bel)
 (den) edge[draw=blue]
   node {} (jail)
   
 (airport1) edge[bend left]
   node {} (bel)
 (airport1) edge[bend left=71]
   node {} (jail)
 (airport1) edge[]
   node {} (neth)
   
 (bel) edge[draw=blue, bend left]
   node {} (jail)
 (bel) edge[draw=blue, bend left=40]
   node {} (neth)
 (bel) edge[draw=blue]
   node {} (airport2)
   
 (jail) edge[bend left]
   node {} (neth)
 (jail) edge[bend left=71]
   node {} (airport2)
 (jail) edge
   node {} (spain)
 
 (neth) edge[draw=blue, bend left]
   node {} (airport2)
 (neth) edge[draw=blue, bend left=40]
   node {} (spain)
 (neth) edge[draw=blue]
   node {} (park)
   
 (airport2) edge[bend left]
   node {} (spain)
 (airport2) edge[bend left=71]
   node {} (park)
 (airport2) edge
   node {} (uk)
   
 (spain) edge[draw=blue, bend left]
   node {} (park)
 (spain) edge[draw=blue, bend left=40]
   node {} (uk)
 (spain) edge[draw=blue]
   node {} (airport3)
   
 (park) edge[bend left]
   node {} (uk)
 (park) edge[bend left=71]
   node {} (airport3)
 (park) edge
   node {} (ita)
   
 (uk) edge[draw=blue, bend left]
   node {} (airport3)
 (uk) edge[draw=blue, bend left=40]
   node {} (ita)
 (uk) edge[draw=blue]
   node {} (gotojail)
   
 (airport3) edge[bend left]
   node {} (ita)
 (airport3) edge[bend left=71]
   node {} (gotojail)
 (airport3) edge
   node {} (fr)
   
 (ita) edge[draw=blue, bend left]
   node {} (gotojail)
 (ita) edge[draw=blue, bend left=40]
   node {} (fr)
 (ita) edge[draw=blue]
   node {} (airport4)
   
 (gotojail) edge
   node {} (jailStrat)
   
  (jailStrat) edge
    node {} (jail)
   
 (fr) edge[bend left]
   node {} (airport4)
 (fr) edge[bend left=40]
   node {} (ger)
 (fr) edge
   node {} (go)
   
 (airport4) edge[bend left]
   node {} (ger)
 (airport4) edge[bend left=71]
   node {} (go)
 (airport4) edge
   node {} (den)
   
 (ger) edge[draw=blue, bend left]
   node {} (go) 
 (ger) edge[draw=blue, bend left=40]
   node {} (den) 
 (ger) edge[draw=blue]
   node {} (airport1);
\end{tikzpicture}
  \end{center}
  \caption{Simplified Markov chain representing the overall board.}
  \label{fig:markov_monop}
\end{figure}

\begin{figure}[H] % Get out of jail markov chain strategies
  % \begin{center}
    \begin{subfigure}[b]{0.5\textwidth}
      \begin{tikzpicture}[font=\small\sffamily]
  \tikzset{my_state/.style={state, 
                                     minimum width=1.3cm,
                                     line width=0.13mm,
                                     fill=gray!20!white}}
  \tikzset{my_state2/.style={state, 
                                     minimum width=1.7cm,
                                     line width=0.43mm,
                                     fill=gray!60!white}}
                                 
  \node[my_state2] (gotojail) {GoToJail}; 
  \node[my_state2, below right = 1 of gotojail] (jail1) {Jail 1};
  \node[my_state2, above right = 1 of jail1] (jail2) {Jail 2};
  \node[my_state2, below right = 1 of jail2] (getOut) {States 2 - 12};
  \node[my_state2, above right = 1 of getOut] (jail3) {Jail 3};
  
  
  
 \draw[every loop, auto=right, line width=0.3mm, >=latex, draw=red, fill=black, bend right]
 
 (gotojail) edge[draw=blue, bend right]
   node {1} (jail1)
   
 (jail1) edge[bend left]
   node {5/6} (jail2)
 (jail1) edge[draw=blue, bend right]
   node {1/6} (getOut)
   
 (jail2) edge[bend left]
   node {5/6} (jail3)
 (jail2) edge[draw=blue, bend right]
   node {1/6} (getOut)
   
  (jail3) edge[draw=blue, bend right]
    node {1} (getOut);
   
\end{tikzpicture}
      \caption{Wait and get double strategy.}
      \label{fig:jail_double}
    \end{subfigure}
    \hspace{3cm}
    \begin{subfigure}[b]{0.24\textwidth}
      \begin{tikzpicture}[font=\small\sffamily]
  \tikzset{my_state/.style={state, 
                                     minimum width=1.3cm,
                                     line width=0.13mm,
                                     fill=gray!20!white}}
  \tikzset{my_state2/.style={state, 
                                     minimum width=1.7cm,
                                     line width=0.43mm,
                                     fill=gray!60!white}}
                                 
  \node[my_state2] (gotojail) {GoToJail}; 
  \node[my_state2, below right = 1 of gotojail] (visit) {Visit};
  
  
 \draw[every loop, auto=right, line width=0.3mm, >=latex, draw=red, fill=black]
 
 (gotojail) edge[draw=blue, bend right]
   node {1} (visit);
   
\end{tikzpicture}

      \caption{Pay fine strategy.}
      \label{fig:jail_pay}
    \end{subfigure}
  % \end{center}
  \caption{Get out of jail strategies.}
\end{figure}

\newpage
% subsection modeling_using_markov_chains (end)


\subsection{Modeling the player \emph{wealth}} % (fold)
\label{sub:modeling_the_player_emph_wealth}

% subsection modeling_the_player_emph_wealth (end)

% section model (end)